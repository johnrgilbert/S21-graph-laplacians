
\par\bigskip
{\bf Problem 3. Conjugate gradient algorithm.}
This is another open-ended problem whose goal is for you to get some
intuition about the behavior of the conjugate gradient algorithm for graph Laplacians.
You are to experiment with CG in Matlab. You may use Matlab's built-in {\tt pcg()}
routine or write your own (simpler) CG in a dozen lines of Matlab.
In the latter case you have the advantage that you can include reorthogonalization
against the constant vector~$w_1$ if necessary 
(though I guess you could add that to Matlab's {\tt pcg.m} source too).
There are two experiments to run:
\par\medskip
{\bf 3(a)}
Experiment with the Laplacian matrices of the same sample graphs you
used for the Lanczos experiments in Homework~2, plus others if you want.
Try CG three ways:  with no preconditioning, with Jacobi preconditioning
(that is, the preconditioner is just the diagonal of the matrix), and
with incomplete Cholesky preconditioning from Matlab's {\tt R = cholinc(L,'0')}.
Make and turn in semilog plots of convergence history of the relative residual
as I did in class on Wednesday, May 7.  Also, for each Laplacian, what is the
finite condition number $\kappa_f(L)$, that is, the ratio of the largest to
smallest nonzero eigenvalue?  Your Lanczos experiments from last time give you
the data to answer this.
\par\medskip
{\bf 3(b)}
Generate two sequences of Laplacian matrices of graphs of increasing sizes as
follows:  The first sequence is the ``model problem'' square grid graph;
Matlab's {\tt L = laplacian(grid5(k));} gives you the $k$-by-$k$ grid with
$n=k^2$ vertices.  The second sequence is flat random graphs, generated with
Matlab's {\tt sprandsym()}.  Set the parameters so that the second sequence
has graphs of about the same sizes and numbers of edges as the first sequence.
{\em Important:} For the second sequence, use just the largest connected 
component of the graph you generate; see Matlab's {\tt components()} routine.
\par
For both sequences, convert the matrices to unweighted Laplacians (with {\tt laplacian()}).
Experiment with CG, both unpreconditioned and preconditioned.  Your goal is to get
enough data to estimate for each sequence the asymptotic scaling of the number of
CG iterations needed to reach some fixed residual reduction 
(say $10^{-6}$) as a function of $n$, the dimension of the matrix.  
Is the scaling different for the two sequences?
Is it different for preconditioned and unpreconditioned CG?
\par
Write a paragraph or so analyzing your results and stating your conclusions.

\end{document}
