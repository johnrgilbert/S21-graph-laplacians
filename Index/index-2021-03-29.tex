\documentclass[11pt]{article}

\usepackage{amsfonts}
\usepackage{amsthm}
\usepackage{amsmath}
\usepackage{graphicx}

\newtheorem{theorem}{Theorem}
\newtheorem*{pathlemma}{Path Lemma}
\newtheorem*{etreelemma}{Etree Fill Lemma}
\newcommand{\m}[1]{{\bf{#1}}}       % for matrices and vectors
\newcommand{\ones}{\m1}             % vector of all ones
\newcommand{\zeros}{\m0}            % vector of all zeros
\newcommand{\diag}{\mbox{diag}}
\newcommand{\dilation}{\mbox{dilation}}
\newcommand{\congestion}{\mbox{congestion}}
\newcommand{\stretchh}{\mbox{stretch}}
\newcommand{\Reff}{R^{\mbox{\scriptsize eff}}}  % effective resistance
%\newcommand{\pinv}{^{\mbox{\scriptsize +}}}    % pseudoinverse
\newcommand{\pinv}{^{\dagger}}                  % pseudoinverse
\newcommand{\Real}{\mathbb{R}}      % real numbers
\newcommand{\krylov}{\mathcal{K}}   % Krylov subspace

\topmargin 0in
\textheight 7.9in
\oddsidemargin 0pt
\evensidemargin 0pt
\textwidth 6.5in

\begin{document}

\title{Index of Notation and Definitions}
\author{CS 292F: Graph Laplacians and Spectra}
\date{Version of March 29, 2021}
\maketitle

There is a lot of variation in terminology and notation in
the field of Laplacian matrix computation and spectral graph 
theory.  
Indeed, even ``Laplacian matrix'' is defined differently by
different authors!

This list gives the versions of notation, terminology, and definitions 
that we will use in CS 292F.
I mostly follow the conventions of Dan Spielman's notes, 
though I prefer not to use greek letters for vectors.
I will keep adding to this list during the quarter.

\begin{enumerate}

\item
Unless otherwise stated, a {\em graph} $G = (V,E)$ is always 
an undirected graph whose $n$ vertices are the integers 
$1$ through $n$, with no multiple edges or loops.

\item
The {\em degree} of a vertex is the number of edges incident on it, 
or equivalently (because we don't allow multiple edges or loops)
the number of its neighboring vertices.

\item 
A graph is said to be {\em regular} if every vertex has the same degree.

\item
A graph is said to be {\em connected} if, for every choice of two
vertices $i$ and $j$, there is a {\em path} of edges from $i$ to $j$.
The {\em connected components} of a graph are its maximal connected
subgraphs.

\item
$K_n$ is the {\em complete graph}, which has $n$ vertices and all $n(n-1)/2$ possible edges.

\item
$P_n$ is the {\em path graph}, which has $n$ vertices and $n-1$ edges in a single path.

\item
$S_n$ is the {\em star graph}, which has $n$ vertices, one with degree $n-1$ and 
$n-1$ with degree 1.

\item
$H_k$ is the {\em hypercube graph}, which has $n=2^k$ vertices, all of degree $k$.
Vertices $i$ and $j$ have an edge between them if $i$ and $j$ differ by a power of 2.
Equivalently, we can identify each vertex with a subset of $\{1,\ldots,k\}$,
with edges to just those subsets formed by adding or deleting one element.

\item
$G_e$ or $G_{(i,j)}$ is the graph with $n$ vertices and only one edge $e = (i,j)$.

\item
We will write a {\em vector} as a lower-case latin letter, 
possibly with a subscript, like $x$ or $w_2$.  
We often think of an $n$-vector as a set of labels for the
$n$ vertices of a graph; 
in that case element $i$ of vector $x$ is written as $x(i)$,
and we may write $x\in\Real^V$ instead of $x\in\Real^n$.
In linear algebraic expressions, vectors are column vectors.

\item
Two special vectors are $\zeros$, the vector of all zeros,
and $\ones$, the vector of all ones.

\item
If $i$ is a vertex then $\ones_i$~is the 
{\em characteristic vector} of~$i$, which
is zero except for $\ones_i(i)=1$.
Similarly if $S$~is a set of vertices, 
then $\ones_S$ is the vector that is equal to one
on the elements of~$S$ and zero elsewhere.

\item
If $d$ is an $n$-vector, $\diag(d)$ is the $n$-by-$n$ diagonal 
matrix with the elements of $d$ on the diagonal.
If $A$ is any $n$-by-$n$ matrix, $\diag(A)$ is the $n$-vector
of the diagonal elements of $A$.

\item\label{lap}
The {\em Laplacian} of graph $G$ is the $n$-by-$n$ matrix $L$
whose diagonal element $L(i,i)$ is the degree of vertex $i$, 
and whose off-diagonal element $L(i,j)$ is~$-1$ if $(i,j) \in E$ 
and $0$ if $(i,j) \notin E$.
This matrix, which we (and Spielman) just call the Laplacian,
is sometimes called the {\em combinatorial Laplacian} to 
distinguish it from the normalized Laplacian 
(to be defined later).
% below~(\ref{nlap}).
Note that $ L\ones = \zeros$.

\item
$L_e$ or $L_{(i,j)}$ is the $n$-by-$n$ Laplacian matrix
of the graph with $n$ vertices and only one edge $e = (i,j)$.
This matrix has only four nonzero elements, two 1's on the
diagonal and two $-1$'s in positions $(i,j)$ and $(j,i)$;
thus 
$$L_{(i,j)}=(\ones_i-\ones_j)(\ones_i-\ones_j)^T.$$
The Laplacian of any graph $G=(V,E)$ is the sum of the Laplacians
of its edges,
$$L_G = \sum_{e\in E} L_e.$$

\item
The {\em Laplacian quadratic form} (or just LQF) is $x^TLx$,
where $L$ is a particular graph's Laplacian and $x$ is a variable $n$-vector.
Its value for a particular vector $x$ is 
$$x^TLx = \sum_{(i,j)\in E}(x(i)-x(j))^2.$$

\end{enumerate}

\end{document}

